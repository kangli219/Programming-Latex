\documentclass[notes,11pt, aspectratio=169]{beamer}

\usepackage{pgfpages}
% These slides also contain speaker notes. You can print just the slides,
% just the notes, or both, depending on the setting below. Comment out the want
% you want.
\setbeameroption{hide notes} % Only slide
%\setbeameroption{show only notes} % Only notes
%\setbeameroption{show notes on second screen=right} % Both


%% Presentation Themes with a Mini Frame Navigation
\usetheme{Frankfurt}  % a variation of Berlin that is slightly less cluttered by leaving out the subsection* information
\setbeamertemplate{footline}[page number]
%\usetheme[compress]{Berlin}
%\usetheme{Singapore}


%% colors
\usecolortheme{seahorse}
%\usecolortheme{rose}
%\usefonttheme{serif}
%\usefonttheme{professionalfonts}


%% special set
\setbeamertemplate{itemize items}[default]
\setbeamertemplate{enumerate items}[default]
\setbeamerfont{title}{size=\large}
\setbeamerfont{frametitle}{size=\large}
\setbeamerfont{subsection* in toc}{size=\small}

\setbeamertemplate{section in toc}{%
	{\color{blue}\inserttocsectionnumber.}~\inserttocsection}

\setbeamertemplate{subsection* in toc}{%
	\hspace{1.5em}{\color{blue}\rule[0.3ex]{3pt}{3pt}}~\inserttocsubsection* \par}

% macros by https://github.com/kangli219/beamer-tips/blob/master/slides.tex
% \setbeamercolor{frametitle}{fg=blue}
% \setbeamercolor{title}{fg=black}
% \setbeamertemplate{footline}[frame number]
% \setbeamertemplate{navigation symbols}{} 
% \setbeamertemplate{itemize items}{-}
% \setbeamercolor{itemize item}{fg=blue}
% \setbeamercolor{itemize subitem}{fg=blue}
% \setbeamercolor{enumerate item}{fg=blue}
% \setbeamercolor{enumerate subitem}{fg=blue}
% \setbeamercolor{button}{bg=MyBackground,fg=blue}

%% make animation of item lists grey rather than disappearing
\setbeamercovered{transparent}
\setbeamertemplate{footline}[frame number]{}

% define color
% These are my colors -- there are many like them, but these ones are mine.
\definecolor{blue}{RGB}{0,114,178}
\definecolor{red}{RGB}{213,94,0}
\definecolor{yellow}{RGB}{240,228,66}
\definecolor{green}{RGB}{0,158,115}

% fonts
\usepackage[default]{lato}  % this is a great font
\usepackage{mathpazo} % math font

% textcolor
\usepackage{xcolor} % text color

% mathematic symbols
\usepackage{amsmath}
\usepackage{amssymb}     
\usepackage{bbm}  % for indicator function

% to change line space
\usepackage{setspace}

% enumerate
\usepackage{enumerate}
\newenvironment{wideitemize}{\itemize\addtolength{\itemsep}{10pt}}{\enditemize}  % have a wider itemize

% input graphs
\usepackage{graphicx}     
% \usepackage{xeCJK}
\usepackage{caption}
\usepackage{subcaption}
\graphicspath{ {image/} }

% draw figures
\usepackage{tikz}

% deal with citations
\usepackage{natbib}  
% different citation styles
%\bibliographystyle{unsrtnat}
\bibliographystyle{apalike}

% make tables
\usepackage{threeparttable, booktabs}
\usepackage{adjustbox}
\usepackage{booktabs}
\usepackage{array} % animate table presentation
                   % see https://tex.stackexchange.com/questions/274920/how-to-uncover-a-table-column-wise-in-latex-beamer
\usepackage{placeins} % place tables in the exact page I want, uisng \FloatBarrier

% hyperlinks
\usepackage{hyperref} 

% renumber pages
\usepackage{appendixnumberbeamer} 




%%%%%%%%%%%%%%%%%%%%%%%%%%%%%%%%%%%%%%%%%%%%%%%%
\title[ ] % (optional, use only with long paper titles)
{title}


\author[] % (optional, use only with lots of authors)
{Kangli Li \inst{1} \and Jordan van Rijn \inst{2}}

\institute[]{\inst{1} University of Wisconsin-Madison \and 
             \inst{2} Credit Union National Association}
             
% \author[PGP]{}
% \institute[FRBNY]{\small{\begin{tabular}{c c c}
% Author A &&  Paul Goldsmith-Pinkham  \\
% Somewhere Fancy && FRBNY \\ \\

% Author C && Author D   \\
% \multicolumn{3}{c}{Somewhere Fancy} \\
% \end{tabular}}}

\date[ ] % (optional, should be abbreviation of conference name)
{Freddie Mac Presentation}



\begin{document}
	\setbeamertemplate{navigation symbols}{}
	% turn off navigation symbols at the the bottom right-hand corner of the screen
	
\begin{frame}
	\titlepage
\end{frame}


%%%%%%%%%%%%%%%%%%%%%%%%%%%%%%%%%%%
%%%%%%%%%%%%%%%%%%%%%%%%%%%%%%%%%%%
%%%%%%%%%%%%%%%%%%%%%%%%%%%%%%%%%%%
%%%%%%%%%%%%%%%%%%%%%%%%%%%%%%%%%%%
\section{Introduction}

%%%%%%%%%%%%%%%%%%%%%%%%%%%%%%%%%%%
%%%%%%%%%%%%%%%%%%%%%%%%%%%%%%%%%%%
\subsection*{Motivation}


%%%%%%%%%%%%%%%%%%%%%%%%%%%%
\begin{frame}{Motivation with listings}

\begin{columns}[T] % align columns
\begin{column}{.45\textwidth}
  something on the left
  \begin{wideitemize}
  \item <1-> listing by order
  	\begin{itemize}
    	\item [-] different starts
  	\end{itemize}
  \item <2-> in the second page
 \end{wideitemize}
\end{column}

\begin{column}{.5\textwidth}
  something on the right
\end{column}

\end{columns}
\end{frame}



%%%%%%%%%%%%%%%%%%%%%%%%%%%%%%%%%%%
%%%%%%%%%%%%%%%%%%%%%%%%%%%%%%%%%%%
\subsection*{Research Questions}


%%%%%%%%%%%%%%%%%%%%%%%%%%%%%%%%%%%
\AtBeginSection[]
{
    \begin{frame}
        \frametitle{Roadmap}
        \tableofcontents[currentsection]
    \end{frame}
}
% 1. using \subsection* to avoid it to show up 
% 2. Another way is to do transitionframe below each section. For example,
% \begin{transitionframe}
%   \begin{center}
%     { \Huge \textcolor{black}{Spacing and Words}}
%   \end{center}
% \end{transitionframe}



%%%%%%%%%%%%%%%%%%%%%%%%%%%%%%%%%%%
%%%%%%%%%%%%%%%%%%%%%%%%%%%%%%%%%%%
%%%%%%%%%%%%%%%%%%%%%%%%%%%%%%%%%%%
%%%%%%%%%%%%%%%%%%%%%%%%%%%%%%%%%%%
\section{Background}

%%%%%%%%%%%%%%%%%%%%%%%%%%%%%%%%%%%
%%%%%%%%%%%%%%%%%%%%%%%%%%%%%%%%%%%
\subsection*{Background about credit unions}


%%%%%%%%%%%%%%%%%%%%%%%%%%%%
\begin{frame}{figure and subfigure}
This frame shows how array figures.
%\begin{figure}
%\centering
%\begin{subfigure}{0.48\textwidth}
%    \includegraphics[width=1\textwidth]{Figures/inst_counts.png}
%    \caption{Institution counts}
%    \label{fig:inst_counts} 
%\end{subfigure}  % with blank line, it aligns subfigures vertically
%\begin{subfigure}{0.48\textwidth}
%    \includegraphics[width=1\textwidth]{Figures/inst_assets_median.png}
%    \caption{Median assets}
%    \label{fig:median_assets}
%\end{subfigure}
%\end{figure}
\end{frame}


\begin{frame}{add tables}
\begin{table}
\centering
\begin{adjustbox}{max width=0.8\textwidth}
\begin{tabular}{l*{6}{c}}
\toprule
&\multicolumn{3}{c}{Bank}&\multicolumn{3}{c}{Credit union}\\
\cmidrule(lr){2-4}\cmidrule(lr){5-7}
&\multicolumn{1}{c}{Proportion}&\multicolumn{1}{c}{S.D.}&\multicolumn{1}{c}{N}&\multicolumn{1}{c}{Proportion}&\multicolumn{1}{c}{S.D.}&\multicolumn{1}{c}{N} \\
\midrule
  \textbf{Panel A: Loan portfolio} & & & & & & \\
  commercial&       0.274&       0.150&       62669&       0.040&       0.069&       29066\\
  real estate&       0.330&       0.214&       62669&       0.481&       0.193&       29066\\
  consumer&       0.051&       0.078&       62669&       0.460&       0.186&       29066\\
  agricultural&       0.069&       0.126&       62669&       0.002&       0.023&       29066\\
  \midrule
  \textbf{Panel B: Mortgage Purpose} & & & & & & \\
  purchase  &       0.443&       0.206&       62513&       0.213&       0.194&       28923\\
  home improvement &       0.098&       0.131&       62513&       0.246&       0.265&       28923\\
  refinance      &       0.414&       0.201&       62513&       0.522&       0.252&       28923\\
  \bottomrule
\end{tabular}
\end{adjustbox}
\end{table}
\end{frame}



%%%%%%%%%%%%%%%%%%%%%%%%%%%%%%%%%%%
%%%%%%%%%%%%%%%%%%%%%%%%%%%%%%%%%%%
%%%%%%%%%%%%%%%%%%%%%%%%%%%%%%%%%%%
%%%%%%%%%%%%%%%%%%%%%%%%%%%%%%%%%%%
\section{Model}

%%%%%%%%%%%%%%%%%%%%%%%%%%%%
\begin{frame}{Conceptual Framework, with beautiful underbrace}

A financial institution optimizes:
\begin{align*}
	& \max_{L^H, L^N}  \:\:  \lambda \underbrace{ B(L^H,L^N, s)}_{\text{\textcolor{red}{consumer utility}}} + (1-\lambda) \underbrace{ \pi(L^H, L^N,s)}_{\text{\textcolor{green}{profit}}} \\
	& \text{ subject to  } \underbrace{L = D + E}_{\text{\textcolor{blue}{balance sheet constraint}}}, \:\:\: \underbrace{L = L^H + L^N}_{\text{\textcolor{blue}{Loans of high and low risk}}} 
\end{align*}
\begin{wideitemize}
    \item <1-> $s \in [0,1]$: state of economy. $s=0$: economy recession
    % \item <1-> $\lambda$ gauges the degree to which consumer's utility is internalized. $\lambda=0$: bank. 
    \item <2-> $B(L^H, L^N, s)= \underbrace{U(L)}_{\text{\textcolor{red}{loan availability}}}-\underbrace{P(L^H, s) V(L^H,s)}_{\text{\textcolor{green}{disutility when default}}}$ 
	\item <3> $\pi(L^H, L^N)  = \underbrace{[1-P(L^H, s)]R^H(s) L^H + R^N L^N}_{\text{\textcolor{red}{loan revenue}}} - \underbrace{R^D D -\Phi(L)}_{\text{\textcolor{green}{deposit and issuance cost}}}$
\end{wideitemize}
\end{frame}




%%%%%%%%%%%%%%%%%%%%%%%%%%%%%%%%%%%
%%%%%%%%%%%%%%%%%%%%%%%%%%%%%%%%%%%
%%%%%%%%%%%%%%%%%%%%%%%%%%%%%%%%%%%
%%%%%%%%%%%%%%%%%%%%%%%%%%%%%%%%%%%
\section{Data}

%%%%%%%%%%%%%%%%%%%%%%%%%%%%%
\begin{frame}{Data}
\end{frame}



%%%%%%%%%%%%%%%%%%%%%%%%%%%%%%%%%%%
%%%%%%%%%%%%%%%%%%%%%%%%%%%%%%%%%%%
%%%%%%%%%%%%%%%%%%%%%%%%%%%%%%%%%%%
%%%%%%%%%%%%%%%%%%%%%%%%%%%%%%%%%%%
\section{Empirical Strategy \& Results}

%%%%%%%%%%%%%%%%%%%%%%%%%%%%%%%%%%%
%%%%%%%%%%%%%%%%%%%%%%%%%%%%%%%%%%%
\subsection*{Regressions on subprime lending}

%%%%%%%%%%%%%%%%%%%%%%%%%%%%%
\begin{frame}{Regressions on subprime lending}
The baseline specification: 
\begin{equation*}
    Y_{i,t} = \underbrace{\beta_1 [ Bank_i \times \mathbbm{1}\{ t \leq 2009 \} ]}_{\text{\textcolor{red}{Null hypothesis: $\beta_1=0$}}} + \beta_2 bank_i + X_{i, 2004}' \gamma + \delta_t + \theta_s + \epsilon_{i,t}
\end{equation*}

\begin{wideitemize}
    \item <1-> $Y_{i,t}$: share of mortgages that are subprime. 
    \begin{itemize}
        \item [-] All mortgage originations.
        \item [-] ``Homogeneous'' mortgage originations: conventional, conforming, 1-4 families, first lien, owner-occupied.
    \end{itemize}
    
    \item <2-> $Bank_i$: bank dummy; $\mathbbm{1}\{ t \leq 2009 \}$: dummy of credit expansion period. 
    
    \item <3> $X_{i, 2004}'$: Covariates in year 2004 (robust to 1-year lags).
    
    \item <3> $\delta_t$ and $\theta_s$: year and state fixed effects.
\end{wideitemize}
\end{frame}

%%%%%%%%%%%%%%%%%%%%%%%%%%%%%
\begin{frame}{tables with columns showing up sequentially}
\begin{table}
\centering
\begin{adjustbox}{max width=0.75\textwidth}
\begin{tabular}{lc<{\onslide<2->}c<{\onslide<3->}cc<{\onslide}}
  \toprule
  &\multicolumn{2}{c}{Subprime Share (\%)}&\multicolumn{2}{c}{Subprime Share (\%)}\\\cmidrule(lr){2-3}\cmidrule(lr){4-5}
  &\multicolumn{1}{c}{All}&\multicolumn{1}{c}{Homogeneous}&\multicolumn{1}{c}{All}&\multicolumn{1}{c}{Homogeneous}\\
  \midrule
  bank $\times$ $\mathbbm{1}\{ Year <=2009 \}$&    7.216***&    5.456***&    7.837***&    5.064***\\
                  &  (0.439)   &  (0.593)   &  (0.441)   &  (0.579)   \\
  \addlinespace
  bank            &    7.756***&    8.727***&            &            \\
                  &  (0.998)   &  (1.247)   &            &            \\
  \addlinespace
  Institution Characteristics & $\times$   & $\times$   &            &            \\
  \addlinespace
  Borrower Characteristics & $\times$   & $\times$   & $\times$   & $\times$   \\
  \addlinespace
  State Controls  & $\times$   & $\times$   & $\times$   & $\times$   \\
  \addlinespace
  State FE        & $\times$   & $\times$   &            &            \\
  \addlinespace
  Institutional FE&            &            & $\times$   & $\times$   \\
  \addlinespace
  Year FE         & $\times$   & $\times$   & $\times$   & $\times$   \\
  \midrule
  $\mathnormal{N}$&    71228   &    63821   &    70962   &    63475   \\
  $\mathnormal{R^2}$&    0.241   &    0.306   &    0.588   &    0.617   \\
  Outcome mean    &   12.912   &   18.124   &   12.916   &   18.127   \\
  \bottomrule
  \end{tabular}
\end{adjustbox}
\end{table}
\end{frame}

%%%%%%%%%%%%%%%%%%%%%%%%%%%%%
\begin{frame}[label=robustness_check_subprime]{Robustness checks: using hyperlinks}
Results are robust to alternative methods, samples, and dependent variables.
\begin{wideitemize}
    \item <1-> Matched sample (by propensity score) \hyperlink{subprime_match}{\beamergotobutton{Results using matched sample}} 
\end{wideitemize}
\end{frame}



%%%%%%%%%%%%%%%%%%%%%%%%%%%%%%%%%%%
%%%%%%%%%%%%%%%%%%%%%%%%%%%%%%%%%%%
%%%%%%%%%%%%%%%%%%%%%%%%%%%%%%%%%%%
%%%%%%%%%%%%%%%%%%%%%%%%%%%%%%%%%%%
\section{Conclusions \& next steps}

%%%%%%%%%%%%%%%%%%%%%%%%%%%%%%%%%%%
\begin{frame}{Conclusions}
\end{frame}

\begin{frame}
\centering
\Large{Thank You!} 
\vspace{2em}

\large{Comments and suggestions \\ kangli.li@wisc.edu}
\end{frame} % to enforce entries in the table of 

%%%%%%%%%%%%%%%%%%%%%%%%%%%%%%%%%%%
\begin{frame}
\end{frame} % to enforce entries in the table of contents



%%%%%%%%%%%%%%%%%%%%%%%%%%%%%%%%%%%%%%%%%
%%%%%%%%%%%%%%%%%%%%%%%%%%%%%%%%%%%%%%%%%
%%%%%%%%%%%%%%%%%%%%%%%%%%%%%%%%%%%%%%%%%
%%%%%%%%%%%%%%%%%%%%%%%%%%%%%%%%%%%%%%%%%
\appendix 

%% subprime
%%%%%%%%%%%%%%%%%%%%%%%%%%%%%%%%%%%
\begin{frame}[label=subprime_match]{hyperlink referenced page with a retun button}
\hfill \hyperlink{robustness_check_subprime}{\beamergotobutton{Return}}
\end{frame}

\end{document}
